\documentclass[11pt]{article}
%%% style file you will need for some commands%%%%%%%%%%%%%%%%%%%%%%
%% aahomework is the style file I have used to typeset many commands, feel free to use them in your solutions.
%% bear in mind that if you need to define your command then you will have to make sure that it is not in conflict to my pre-defined command. Otherwise you will need to either use
%% commands defined by me or edit the style file appropriately.
\usepackage{aahomework}
\usepackage{tikz}
\usepackage{tcolorbox}

\newcommand*\circled[1]{\tikz[baseline=(char.base)]{
            \node[shape=circle,draw,inner sep=2pt] (char) {#1};}}
%%%the \circled command has been used to create text inside circle for grading table.

%%%\geometry{letterpaper, textwidth=17cm, textheight=22cm}

%%%%%%%%%%%%%%%%%%%%%%%%%%%%%%%%%%%% the following is for the cover sheet--FILL IN appropriately%%
\newcommand{\mycourse}{Introduction to Computer Linguistics 8.1008}
\newcommand{\semesteryear}{Spring 2016}
\newcommand{\myname}{Timothy Fairman, Kaitlin Hipkin, Tyler Wilgenbusch}  %%<<<<<<<<<<<<<<<|========================================= (please put your name here)==========
\newcommand{\hwnumber}{2} %%<<<<<<<<<<<<<<<|========================================= (please put HW number here, e.g. 1,2,3...)==========

%%%%%%%%%%%%%%%%%%%%%%%%%%%%%%%%%%%%%% following is NOT to be edited, DO NOT type anything here, it will receive inputs from what you fill above%%%%%%%%%%%%%%%%%%
\title{Homework \hwnumber} %% DO NOT type in HW number here
\author{\myname} %% DO NOT type in your name here.
\date{\textbf{\mycourse} \hfill {\today} \hfill \textbf{\semesteryear}} %% DO NOT TYPE in mycourse and/or quarteryear values
%%%%%%%%%%%%%%%%%%%%%%%%%%%%%%%%%%%%%%%%%%%%%%%%%%%%%%%%%%%%%%%%%%%%%%%%%%%%%%%%%%%%%%%%%%%%%%%%%%%%%%%%%%%%%%%%%%%%%%%%%%%%%%%%

\setlength{\parindent}{0pt} %% paragraphs will not be indented
\setlength{\parskip}{.25cm} %% space between paragraphs
\linespread{1.1}

\begin{document}
\thispagestyle{empty} %%this is to supress the page number on the cover page

\clearpage %% these are to reset the page number for the first page of your homework to 1.
\pagenumbering{arabic} %% these are to reset the page number for the first page of your homework to 1.
\maketitle

%%%%%%%%%%%%%%%%%%%%%%%%%%%%%%%%% you may start typing below%%%%%%%%%%%%%%%%%%%%%%%%%%%%%%%%%%%%%%%
%% In my style file aahomework.sty I have defined two environments "problem" and "solution" that can be used to type in your question and answer respectively as shown below.%%

\begin{problem}{1}
\textbf{Morphological Parsing}

Give a morphological analysis of the following words,
\begin{tcolorbox}
	\textbf{English}: printer, underdetermination, incremental \\
	\textbf{German}: Versprecher, Ansprechpartner, übergezogener
\end{tcolorbox}

The morphological analysis should be broken down as follows:
\begin{itemize}
	\item Say of each simplex morpheme whether it is a stem, a base, a root, a prefix, a derivational, or an inflectional suffix.
	\item Give the order of composition, indicating at each level which morpheme is the base.
	\item Mark each base for its POS.

\end{itemize}


Example: \textbf{unabsehbar}
\begin{itemize}
	\item Verbal Root (=base): \textbf{seh-} combines with derivational prefix \textbf{ab-} to yield complex \textit{verbal base} \textbf{abseh-}

	\item Verbal Base: \textbf{abseh-} combines with derivational suffix \textbf{-bar} to yield adjective (or \textit{adjectival base}) \textbf{absehbar}

	\item Adjectival Base: \textbf{absehbar} combines with prefix \textbf{un-} to yield \textit{adjective} \textbf{unabsehbar}. 
\end{itemize}

\end{problem}

\begin{solution}
Solution for Problem \#1

NOTE: Could probably add an image associated with each breakdown of each word (might be a bit overkill)
\end{solution}

\vspace*{0.5cm} %% this is to put some vertical space betwen the next problem and the previous solution. You can change the value to something more appropriate.

\begin{problem}{2}
\textbf{Plurals}

Is the English plural formation really completely regular? List the allomorphs of the regular English plural morpheme and give three plural forms that are not formed regularly.
\end{problem}

\begin{solution}
Solution for Problem \#2
\end{solution}

\vspace*{0.5cm}

\begin{problem}{3}
\textbf{Words}

Give short definitions (at most 2 lines of text each) for the following terms:
\begin{itemize}
	\item Word form
	\item Grammatical word
	\item Lexeme
\end{itemize}

\end{problem}

\begin{solution}
Solution for Problem \#3
\end{solution}

\vspace*{0.5cm}

\begin{problem}{4}
\textbf{Productivity}

Describe briefly the relation between productivity, rule, and blocking in morphology (at most
5 lines of text).

\end{problem}

\begin{solution}
Solution for Problem \#4
\end{solution}

\vspace*{0.5cm}

\begin{problem}{5}
\textbf{Forms of Derivation}

Give 2 German and 2 English examples each for the following types of derivation:
\begin{itemize}
	\item Compounding
	\item Conversion
	\item Clipping
\end{itemize}

Do \textbf{not} use examples from the lecture slides.

\end{problem}

\begin{solution}
Solution for Problem \#5
\end{solution}

\vspace*{0.5cm}

\begin{problem}{6}
\textbf{X-language Verbal Morphology}

X-language is a variant of a language spoken somewhere in Africa. As the following example shows, X-language's verbs are so complex that they can sometimes convey the meaning of an entire sentence on their own.

\begin{tabular}{l l | l l} 
\hline atalopenda & ``he will like me'' & atalopiga & ``he will beat me'' \\
atakupenda & ``he will like you'' & atakupiga & ``he will beat you'' \\
atampenda & ``he will like him'' & atampiga & ``he will beat him'' \\
atanipenda & ``he will like us'' & analopiga & ``he is beating me'' \\
atawapenda & ``he will like them'' & anakupiga & ``he is beating you'' \\
lotakupenda & ``I will like you'' & anampiga & ``he is beating him'' \\
lotampenda & ``I will like him'' & amekupiga & ``he has beaten you'' \\
lotawapenda & ``I will like them'' & amelopiga & ``he has beaten me'' \\
utalopenda & ``you will like me'' & amempiga & ``he has beaten him'' \\
utampenda & ``you will like him'' & alilopiga & ``he beat me'' \\
nitampenda & ``we will like him'' & alikupiga & ``he beat you'' \\
watampenda & ``they will like him'' & alimpiga & ``he beat him'' \\
atakusumbau & ``he will annoy you'' & wamenilipa & ``they have paid us'' \\
unamsumbau & ``you are annoying him'' & nilikulipa & ``we paid you'' \\ \hline
\end{tabular}

\begin{description}
	\item[a.] Give the X-language morphemes corresponding to the following English morphemes: I, me, us, you (as subject), you (as object), he, him, they, them, -ed (past tense), will (future), is+-ing (present progressive), has+ -en (present perfect), and the verbal roots like, annoy, beat, and pay.
	\item[b.] Illustrate how the morphemes combine by giving all inflected forms of the X-language equivalent of the verb annoy for past and future in the 1 st , 2 nd , and 3 rd person singular and plural subject and 2 nd person object, with their translation into English.
\end{description}

\end{problem}

\begin{solution}
Solution for Problem \#6
\end{solution}

\vspace*{0.5cm}

\begin{problem}{7}
\textbf{Two-level Morphology for Regular English Past Tense.}

Suppose the grammatical form for the past tense of the English verb \underline{gaze} is $<$gaze,+V,+Past$>$ and for the verb \underline{walk} $<$walk,+V,+Past$>$, etc. while the surface forms are \textit{gazed} and \textit{walked}.

\begin{description}
	\item[a.] What would be the \textbf{underlying form} (showing morpheme and word boundaries) for these two past tense forms?
	
	\item[b.] What could an \textbf{FST for the lexical mapping} look like, i.e., an FST that transforms the grammatical form into the underlying form for all regular English verbs? Give the FST in the form of a graph, as in the lecture slides for the plural English FST.
	
	\item[c.] What could an \textbf{FST for the surface mapping} look like, i.e., an FST that transforms the underlying form into the surface form for all regular English verbs? Give the FST in the form of a graph, as in the lecture slides for the plural English FST.
\end{description}

Hint: The regular English past tense is formed by adding –ed to the base, unless the base ends in e; in the latter case only –d is added.

\end{problem}

\begin{solution}
Solution for Problem \#7
\end{solution}

\end{document}